\chapter{Introduction} \label{cha:introduction}

% Motivation
\section{Motivation} \label{sec:motivation}
As the years go by, new and improved technologies are created and introduced into the world. This progress leads to a number of new ideas that can be implemented. One of these ideas is to automate different monotonous tasks and for this, small robots have been developed. The iRobot Roomba vacuum cleaner~\citep{roomba}, is a robot already well known for being able to perform the, for a human, menial task of vacuum cleaning the house. Similar robots can be found, that can perform similar tasks, such as mowing the lawn or moving goods in a warehouse. Another concept of a robotic helper could be an \textbf{A}utonomous \textbf{R}obot \textbf{C}ollector (\projname{}), a robot which can move around in a limited environment, while collecting small objects in an efficient way. This idea is the basis for this project.

The \projname{}-system stems from a concept, that can be applied to solve a number of menial tasks. As mentioned before, the Roomba can vacuum the floor, but is unable to remove any items which do not belong on the floor, for example toys in a child's bedroom. The \projname{}-system could, potentially, be able to remove items from the floor in a room so that it can be vacuumed, or perhaps collect trash on the sidewalks and roads. The \projname{} could even be combined with the vacuum cleaner, by removing big objects in the path, allowing for the spot to be vacuum cleaned. It could also be used to distribute parcels in an office or warehouse.

Another applicable area could be after an event, such as a concert, where the \projname{} has to remove trash to prepare for a new event. A robot which can perform such tasks, while being efficient within a reasonable time, can provide improvements in a number of industries and environments, and with that in mind, the \projname{} has been developed in this project as a proof of concept. The \projname{}-prototype works inside a specified environment, and collect objects that suit its collection assignment. 

The \projname{} is an intelligent embedded system. The common characteristic of an intelligent system is the robot's reasoning, based on an adequate representation of the world with enough information to allow algorithmic solution procedures to be defined. The \projname{} should be able to make intelligent decisions based on the currently perceived state of the environment.

Embedded systems make up a large percentage of all electronic systems in the world(approximately 90\%~\citep{embedded_software_number}). Embedded software can be anything from refrigerators to small wrist watches to intelligent software in modern cars. When developing embedded software some boundaries exist, as the small computers have a limited amount of computational power. These limitations have been taken into account in the design and development of the \projname{} system. 



%Embedded systems make up a large percentage of all electronic systems in the world(approximately 90\%~\citep{embedded_software_number}). Embedded software can be anything from refrigerators to small wrist watches to intelligent software in modern cars. When developing embedded software some boundaries exist, as the small computers have a limited amount of computational power. This must be considered when developing software, as some tasks may require more time to compute than others, which in turn may result in some tasks never being computed. This is especially important in real-time systems, as some tasks may be very important, to ensure that the system functions properly. These limitations have been taken into account in the design and development of the \projname{} system. 


% Physical Requirement
\section{Physical requirements} \label{sec:physical_requirements}
This sections describes the physical requirements that the \projname{} should be able to accomplish. The system's requirements are reflections of the real-life requirements that would allow a similar system to function properly. 

\subsection{Driving}
In order to move around the environment, the robot must be able to drive itself forward and backwards. This is an obvious requirement, however there are numerous non-obvious challenges associated to this. The wheels used must be able to support the frame of the robot while keeping it stable, as well as being able to move straight at an equal speed on a number of surfaces with different friction.

\subsection{Braking}
A requirement of \projname{} is that it should be able to stay within a specified environment. To complete this goal, the robot must be able to brake, to keep itself from overstepping the boundary. Furthermore, as the \projname{} encounters an object, it should be able to stop itself within the correct distance to be able to collect it.

\subsection{Turning}
If the robot is to explore the entirety of the environment, it must be able to turn. It should also be able to turn with at least some precision, since this allows the robot to make reasonable assumptions about its position relative to the environment. 

\subsection{Collecting unit and storage}
The robot must be able to collect objects within a given environment. When a collectible object is encountered, the robot must stop to collect it. The collecting unit picks up the object and the robot stores it within itself and continue to search for more objects to collect. This presents some challenges for the construction of the collecting unit and the robots storage capacity.

The collection unit must be strong enough to pick up objects and place these in the robots storage. The collectible objects can be of varying sizes, which the collection unit must also support. It is required that the collection unit must be able to support picking up objects with a weight of maximum 150-200g and dimensions of 6x6x6cm for square objects and a diameter of 6-7cm for cylinder objects.\fxnote{hvor har vi de tal og vægte fra?}

Another requirement is for the robot to have some place to store the collected objects. The storage is located on the robot itself. The storage capacity of the robot must be sufficiently large to contain multiple objects of the given dimensions. The robot must furthermore be solid enough to support the extra weight of the collected objects. 

%In this project, this environment consists of a flat plane, with the boundaries marked.
%It should also be able to somehow get itself back inside the environment, should it accidentally leave it.
%If it cannot turn precisely, calculating positions in the environment gets hard.
%such that the collecting unit places the picked up objects within this storage, and the robot drives around the environment with the collected objects. 
%Also, if an object is detected in a position in which the robot cannot immediately collect it, it should be able to manoeuvre itself closer and collect it.


% Problem statement
\section{Problem statement} \label{sec:problem_statement}
Based on the motivation from \secref{sec:motivation} and the physical requirements from \secref{sec:physical_requirements}, this project focuses on the following problem:

\begin{center}
\textit{How is it possible to develop an autonomous robot which can locate all objects in an environment, collect the objects from the ground and place these objects in a storage?}
\end{center}

\subsection{Delimitation} \label{sec:delimitation}
The robot developed in this project is a prototype of the concept mentioned in \secref{sec:motivation}. It is not a finished product, that can fulfil the ambition to automate the aforementioned tasks, but a proof of concept. It is a platform that can be expanded upon in the future. This means, that in order make the implementation of \projname{} feasible, certain constraints have been defined. This results in a controlled environment with controlled parameters, where it is more suitable to develop and test a prototype.

\begin{itemize}
\item The environment is a flat plane outlined by black lines that form a convex set of no more than 2 square meters.
\item The robot can only operate with a specified size and shape of objects.
\item The robot does not discriminate between objects.
\end{itemize}


%This project concern itself with developing an autonomous robot collector. Given a number of objects to find, the robot should be able to navigate an environment. It should be able to explore all of the environment, looking for objects, until it has found all of these. When it finds an object, it should be able to pick it up utilising a collection unit, and dump the objects in a storage.


% System definition
\section{System definition} \label{sec:system_definition}

With the project focus stated into a problem statement and delimited to a prototype size, the desired system can now be defined. The system definition of the \projname{} is the following:

\begin{table}[H]
	\centering
	\begin{narrow}{1cm}{1cm}

	\rule{\linewidth}{0.035cm}
	\begin{center}
	\textbf{System Definition} 
	\end{center}
	\rule{\linewidth}{0.035cm}
    
	\medskip\noindent \textit{Given a number of objects to find, the robot should be able to navigate an environment, defined by black lines, while searching for these objects. While doing this, the robot picks up objects it detects and stores them within its storage container.}

	\rule{\linewidth}{0.035cm}
    \label{sec:systemdefinition}
	\end{narrow}
\end{table}




















%As the years go by, new and improved technologies are created and introduced into the world. This progress leads to a number of new ideas, that can be implemented into the world. One of these ideas, is the idea to have self-driving cars, that are able to drive around safely and without interaction from the users. Autonomous cars have already been developed and are being used on the roads. As an example Google has made a working self-driving car~\citep{GoogleCar}. If cars can be made, the possibilities to create a wide range of usable vehicles like ambulances, public transportation, delivery cars and trucks can be developed as well. 

%Autonomous cars can improve the traffic in a number of different ways, if it was an ideal software, that was implemented on all vehicles. First of all, all accidents and crashes could be avoided, if there was some system that made sure that every car followed the rules. This would make the traffic run smoother, and make it more efficient overall. 

%On the smaller scale, robot are being used in private houses and in the industrial sector. Robots can be used 

%This project concern itself with developing an autonomous robot collector. Given a number of objects, the robot should be able to navigate an environment, marked with black tape, by locating the edges and decide that it should not move outside of these. It should be able to explore all of the environment, looking for objects, until it has found all of these. When it finds an object, it should be able to pick it up utilising a crane-claw, and dump the objects in a storage container on the back of the robots frame.