\section{Motivation} \label{sec:motivation}
As the years go by, new and improved technologies are created and introduced into the world. This progress leads to a number of new ideas that can be implemented. One of these ideas is to automate different monotonous tasks and for this, small robots have been developed. The iRobot Roomba vacuum cleaner~\citep{roomba}, is a robot already well known for being able to perform the, for a human, menial task of vacuum cleaning the house. Similar robots can be found, that can perform similar tasks, such as mowing the lawn or moving goods in a warehouse. Another concept of a robotic helper could be an \textbf{A}utonomous \textbf{R}obot \textbf{C}ollector (\projname{}), a robot which can move around in a limited environment, while collecting small objects in an efficient way. This idea is the basis for this project.

The \projname{}-system stems from a concept, that can be applied to solve a number of menial tasks. As mentioned before, the Roomba can vacuum the floor, but is unable to remove any items which do not belong on the floor, for example toys in a child's bedroom. The \projname{}-system could, potentially, be able to remove items from the floor in a room so that it can be vacuumed, or perhaps collect trash on the sidewalks and roads. The \projname{} could even be combined with the vacuum cleaner, by removing big objects in the path, allowing for the spot to be vacuum cleaned. It could also be used to distribute parcels in an office or warehouse.

Another applicable area could be after an event, such as a concert, where the \projname{} has to remove trash to prepare for a new event. A robot which can perform such tasks, while being efficient within a reasonable time, can provide improvements in a number of industries and environments, and with that in mind, the \projname{} has been developed in this project as a proof of concept. The \projname{}-prototype works inside a specified environment, and collect objects that suit its collection assignment. 

The \projname{} is an intelligent embedded system. The common characteristic of an intelligent system is the robot's reasoning, based on an adequate representation of the world with enough information to allow algorithmic solution procedures to be defined. The \projname{} should be able to make intelligent decisions based on the currently perceived state of the environment.

Embedded systems make up a large percentage of all electronic systems in the world(approximately 90\%~\citep{embedded_software_number}). Embedded software can be anything from refrigerators to small wrist watches to intelligent software in modern cars. When developing embedded software some boundaries exist, as the small computers have a limited amount of computational power. These limitations have been taken into account in the design and development of the \projname{} system. 



%Embedded systems make up a large percentage of all electronic systems in the world(approximately 90\%~\citep{embedded_software_number}). Embedded software can be anything from refrigerators to small wrist watches to intelligent software in modern cars. When developing embedded software some boundaries exist, as the small computers have a limited amount of computational power. This must be considered when developing software, as some tasks may require more time to compute than others, which in turn may result in some tasks never being computed. This is especially important in real-time systems, as some tasks may be very important, to ensure that the system functions properly. These limitations have been taken into account in the design and development of the \projname{} system. 
