\section{Problem statement} \label{sec:problem_statement}
Based on the motivation from \secref{sec:motivation} and the physical requirements from \secref{sec:physical_requirements}, this project focuses on the following problem:

\begin{center}
\textit{How is it possible to develop an autonomous robot which can locate all objects in an environment, collect the objects from the ground and place these objects in a storage?}
\end{center}

\subsection{Delimitation} \label{sec:delimitation}
The robot developed in this project is a prototype of the concept mentioned in \secref{sec:motivation}. It is not a finished product, that can fulfil the ambition to automate the aforementioned tasks, but a proof of concept. It is a platform that can be expanded upon in the future. This means, that in order make the implementation of \projname{} feasible, certain constraints have been defined. This results in a controlled environment with controlled parameters, where it is more suitable to develop and test a prototype.

\begin{itemize}
\item The environment is a flat plane outlined by black lines that form a convex set of no more than 2 square meters.
\item The robot can only operate with a specified size and shape of objects.
\item The robot does not discriminate between objects.
\end{itemize}


%This project concern itself with developing an autonomous robot collector. Given a number of objects to find, the robot should be able to navigate an environment. It should be able to explore all of the environment, looking for objects, until it has found all of these. When it finds an object, it should be able to pick it up utilising a collection unit, and dump the objects in a storage.
