\section{Platform} \label{sec:platform}
To build the \projname{}, a platform has to be chosen. There exist multiple platforms for which it is possible to develop an autonomous robot such as the \projname{}. This section looks into some of the most accessible platforms to consider when creating an embedded system. The chosen platform is further described along with its specifications.

\subsection{Arduino}
The first platform considered was the Arduino. The Arduino single-board micro controller is an open-source hardware platform, intended to make the creation of small, interactable devices accessible to students. As such, it has a fairly low price. The Arduino comes with a fully equipped Integrated Development Environment (IDE), and can be programmed in either C or C++~\citep{arduino}.

\subsection{Raspberry Pi}
Another platform that was considered for the \projname{} was the Raspberry Pi. The Raspberry Pi is developed by the Raspberry Pi Foundation as a single board computer, the size of a credit card. It is a microcomputer, which is able to perform various tasks, such as word processing and video transmitting. The specifications vary depending on the model. Model A, has a 700MHz CPU, 256MB memory and needs a SD card for storage. The operating system(OS) is based on Linux and two of the available OSes are Raspbian and Arch Linux~\citep{raspberry_pi}. 

\subsection{Lego NXT} \label{sec:lego_nxt}
The last considered platform was the LEGO Mindstorm NXT 2.0. It consists of a series of sensors and actuators that are all connected to a programmable NXT Intelligent Brick(NXT Brick). The construction itself can be built with bricks from the LEGO Technic series. The primary component in the Mindstorm NXT series is the NXT Brick, which is used to control the components connected the it. The NXT Brick has four input sockets which can be used for various sensors and three output sockets which are used to send signals to the motors that controls the robot~\citep{lego_nxt_2.0}.

\subsection{Choice of platform}
All of these three platforms are good choices for creating an embedded system. Each of them have their positive and negative aspects, but in regards to the requirements of the \projname{} it was been decided that the Lego NXT is the most suitable platform for this project.

LEGO NXT includes motors and sensors built for the NXT. All sensors and motors as well as the NXT Brick itself are constructed in such a way, that they are straightforward to assemble with the LEGO Technic parts. 

Furthermore, LEGO NXT has been widely used for building robotic projects in the faculty and it was therefore the natural choice. The actual construction of the body of the robot is more cumbersome with the Arduino or Raspberry Pi. Getting the electronics working requires a lot of wiring and soldering while building the actual body might require welding the parts together and generally put a lot more emphasis on the physical components. New device drivers would also have to be written for the sensors and motors to function. This would require effort and time, outside the scope of the project, and by choosing the LEGO NXT this extra time can be used to program the \projname{}.



