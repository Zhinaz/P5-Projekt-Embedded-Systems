\section{Navigation problem} \label{sec:problem-description}

The problem that the \projname{}-prototype has to solve, is to collect a number of objects within a given environment. It has to do this by driving to and collecting every object, after they have been located. The issue of locating the objects is complex, and the question of how the robot can ensure that all objects in the environment have been found, is hard to answer. To do this, one would have to be able to confirm, whether or not all locations where an object could be, have been checked for an object by a sensor.

Once all objects have been found, the problem is to find a path between all the objects, that the robot can then follow in order to pick them up. The locations of objects could be imagined as a graph, where the vertices are objects. The initial position of the robot would also be a vertex, in the centre of the graph. In this graph, all vertices are connected by edges, that is, every different pair of vertices has an edge between them. This is called a \emph{complete graph}~\citep{hamilton}.

The path needed is known as a Hamiltonian path~\citep{sipser}; it must visit all vertices (objects) in the graph exactly once. Normally, determining whether or not there is a Hamilton path is a very complex problem, however because the graph in this case is a complete graph, every possible path only visiting unvisited vertices is a Hamilton path~\citep{hamilton}. Rather, the problem is to find an acceptable path that, preferably, is as close to the optimal path as possible.

The optimal path between the objects would be the path that is most efficient. Multiple variables affect the efficiency, for instance power used, computations required, or distance travelled. However in this project, the focus will be to reduce the time spent from start to completion. In a final version of the product, you would want the \projname{} to be reasonably fast at cleaning up an area, or at completing any other given task, so that it makes sense to use it instead of doing the task yourself.

To achieve a faster completion time, reducing distance travelled to a reasonable level compared to the shortest possible Hamiltonian path, will be the focus of this report. This problem is known as the \emph{travelling salesman problem}, and is NP-hard \citep{tsp}. To find the shortest route, one would have to calculate the distance of every possible route between the objects, which has a calculation complexity of $\mathcal{O}(n!)$. This is obviously not feasible, and thus heuristics will be used in the following chapters to develop algorithms that can solve the problem within an acceptable time limit, after the accessible hardware has been analysed. 

