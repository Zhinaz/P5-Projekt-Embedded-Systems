\section{Operating system and programming language} \label{sec:os_and_proglanguage}
Since the LEGO Mindstorm NXT was chosen as the platform for the prototype robot, it puts some limitations on the choice of OS and programming language. It is required to use an OS that is supported by the NXT Intelligent Brick and the programming language must furthermore be supported by the chosen OS.

A few different OSes were considered for the \projname{}. The default NXT firmware IDE allows for simple drag-and-drop programming in the accompanying LEGO NXT-G software, for creating simple sample robots quickly. Another IDE called BricxCC, also used with the default firmware, uses the programming language Not eXactly C (NXC) and Next Byte Codes (NBC). The default firmware allows for programming both simple and more advanced robots. 

A considered custom firmware was leJOS NXJ, that uses the programming language Java. The leJOS NXJ OS contains a tiny Java virtual machine to execute the code \citep{lejos}. This implies that, had this OS been chosen, the source code would be written in Java.

Another of the considered custom firmwares is nxtOSEK. nxtOSEK is a hybrid between the device drivers of leJOS NXJ, the TOPPERS/ATK Kernel, and the TOPPERS/JSP Real-Time OS. It supports programming in ANSI C/C++ using GCC (GNU ARM) and contains a C and C++ API for the NXT sensors and motors~\citep{nxtosek, toppers_atk, toppers_jsp, nxtOSEK2, nxtosek_api}.

nxtOSEK supports multithreading and real-time multi tasking features, and more importantly for the \projname{}, events are also supported. Bluetooth connection is supported, both brick to brick, and brick to pc, where R/C and Data Logging possibilities exist.

The default OS is more restrictive than some of the alternatives. Because of this, it was decided to use one of the custom firmwares. The C/C++ languages were favoured over Java and is the main reason why nxtOSEK was chosen over leJOS.

